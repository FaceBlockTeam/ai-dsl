%% Technical Report for the work on the AI-DSL over the period of
%% March to May 2021.

\documentclass[]{report}
\usepackage{url}
\usepackage{minted}
\usepackage[textsize=footnotesize]{todonotes}
\newcommand{\kabir}[2][]{\todo[color=yellow,author=kabir, #1]{#2}}
\newcommand{\nil}[2][]{\todo[color=purple,author=nil, #1]{#2}}
\usepackage[hyperindex,breaklinks]{hyperref}
\usepackage{breakurl}
\usepackage{listings}
\lstset{basicstyle=\ttfamily\footnotesize,breaklines=false,frame=single}
\usepackage{float}
\restylefloat{table}
\usepackage{longtable}
\usepackage{graphicx}
\usepackage[font=small,labelfont=bf]{caption}
\usepackage[skip=0pt]{subcaption}
\usepackage{circledsteps}

\begin{document}

\title{AI-DSL Technical Report (May to Septembre 2022)}
\author{Nil Geisweiller, Samuel Roberti}
\maketitle

\begin{abstract}
\end{abstract}

\tableofcontents

\chapter{Introduction}

\chapter{Conclusion}

\appendix
\chapter{Glossary}
\begin{itemize}
\item \textbf{AI service assemblage}: collection of AI services
  interacting together to fulfill a given function.  Example of such
  AI service assemblage would be the Nunet Fake News Warning system.
\item \textbf{Dependent Types}: types depending on values.  Instead of
  being limited to constants such as \texttt{Integer} or
  \texttt{String}, dependent types are essentially functions that take
  values and return types.  A dependent type is usually expressed as a
  term containing free variables.  An example of dependent type is
  \texttt{Vect n a}, representing the class of vectors containing
  \texttt{n} elements of type \texttt{a}.
\item \textbf{Dependently Typed Language}: functional programming
  language using dependent types.  Examples of such languages are
  Idris, AGDA and Coq.
\end{itemize}

\bibliographystyle{splncs04}
\bibliography{local}

\end{document}
